%--------------------------------------
% Streszczenie po polsku
%--------------------------------------
\clearpage % Zaczynamy od nieparzystej strony
\streszczenie Projekt ma na celu wytworzenie systemu wspierającego zarządzanie osiedlami mieszkaniowymi. W trakcie wytwarzania projektu zostały rozważone różne metody konteneryzacji oraz wdrażania projektu. Ostatecznie projekt został skonteneryzowany za pomocą Kubernetes'a, natomiast potok jego wdrożenia został wykonany za pomocą github actions. Została również wytworzona część monitoringu działania projektu za pomocą bazy elasticsearch, filebeat'u, logstash'a oraz kibana'y. Aplikacja API została wykonana za pomocą Pythona, natomiast interfejs użytkownika za pomocą Typescript'u, React'a oraz Vite. Użyta została również baza postgresql a komunikacja z nią odbywa się za pomocą ORM'u - SQLAlchemy. Zastosowano  również synchronizację między bazą postgresql a elasticsearch za pomocą pg\_sync.
\slowakluczowe konteneryzacja, Kubernetes, ELK, osiedla mieszkaniowe, zarządzanie

%--------------------------------------
% Streszczenie po angielsku
%--------------------------------------
\newpage
\abstract The project aims to create a system supporting the management of housing estates. During the project's development, various methods of containerization and project implementation were considered. Ultimately, the project was containerized using Kubernetes and its implementation pipeline was made using GitHub actions. A part of the project's operation monitoring was also created using the elasticsearch, filebeat, logstash and kibana databases. The API application was developed using Python, while the user interface was created using Typescript, React and Vite. A Postgresql database is also used and communication with it is carried out using an ORM - sqlalchemy. There is also the synchronization between the Postgresql database and elasticsearch using pg\_sync
\keywords containerization, Kubernetes, ELK, housing estates, management