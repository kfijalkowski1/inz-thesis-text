\newpage
\section{Testowanie}
W ramach projektu zostały wykonane 2 typy testów: jednostkowe oraz manualne. Aplikacja EMS-API była testowana z użyciem testów jednostkowych, natomiast interfejs użytkownika z użyciem testów manualnych. Istnieją narzędzia umożliwiające testowanie interfejsów internetowych, takie jak: selenium web driver, czy Playwrite, aczkolwiek w tak wczesnej fazie projektu, przy częstych zmianach interfejsu, dogodniejsze jest zastosowanie testów manualnych. Czas i zasoby poświęcone na implementację poprawnych testów interfejsu mogłyby zająć o wiele dłużej, niż czas poświęcony na regularne testowanie manualne. Testy manualne interfejsu użytkownika były wykonywane zgodnie z instrukcją interfejsu użytkownika.
\subsection{Implementacja testów jednostkowych}
Testy jednostkowe zostały wykonane za pomocą biblioteki pytest-sqlalchemy-mock. Biblioteka ta mockuje połączenie do bazy danych w SQLAlchemy używając technologii pytest oraz fixtures. Dzięki temu działaniu, w momencie pisania testu jednostkowego, do funkcji przekazywany jest argument "mocked\_session", dzięki któremu wykonujemy funkcje na zmockowanej bazie. Relacje i dane początkowe w bazie zdefiniowane są za pomocą słownika, którego fragment znajduje się poniżej.
\begin{minted}{python}
mock_db = {
  "estate": [
      {
          "id": 1,
          "name": "Test Estate",
          "description": "Test Estate Description"
      }...
  ],
  "users": [
      { ...
\end{minted}
Należy zauważyć, że rozwiązanie to posiada pewne ograniczenia, mianowicie, aby mockować bazę, biblioteka ta wykorzystuje sql-light, który nie wspiera wszystkich typów danych, wspieranych natomiast w bazach Postgres. Takim przykładem jest typ UUID, który używany jest w projekcie do dodawania nowych obiektów w bazie. Poniżej znajduje się przykładowy test.
\begin{minted}{python}
def test_get_post(mocked_session):
    post = get_post(mocked_session, "1")
    assert post.title == "Test Post"
    assert post.description == "Test Post Description"
    assert post.author_id == "1"
\end{minted}
