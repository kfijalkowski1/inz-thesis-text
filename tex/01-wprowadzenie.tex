\newpage % Rozdziały zaczynamy od nowej strony.
\section{Wprowadzenie}
\subsection{Cel pracy}
Celem niniejszej pracy było stworzenie systemu mającego na celu kompleksowe wsparcie administracji osiedla mieszkaniowego w najważniejszych obszarach działania, uwzględniając zarówno kwestie dotyczące obsługi mieszkańców, jak i zarządzania pracownikami spółdzielni. 
Jako jeden z wymienionych wyżej obszarów działania administracji osiedla zdefiniowano zagadnienie zgłaszania szkód, usterek i działań konserwacyjnych, zarządzanie zleceniami i rozliczeniem ich naprawy przy użyciu istniejących zasobów ludzkich. Przyjęto założenie, że projektowany system powinien więc pozwalać mieszkańcom na szybkie i wygodne zgłaszanie usterek do administracji osiedla mieszkaniowego, jak również  posiadać łatwo dostępne forum służące do informowania mieszkańców np. o awariach czy planowanych naprawach lub przekazywania ogólnej korespondencji dotyczącej budynku. 
Inną perspektywą, uwzględnioną w projektowanym rozwiązaniu, jest zarządzanie pracą osób wykonujących zlecenia dla spółdzielni, zaprojektowano rozwiązania umożliwiające przypisywanie pracowników do realizacji konkretnych zadań, jak i dodawanie nowych współpracujących osób i określanie ich ról. 
Przy projektowaniu systemu wzięto również pod uwagę taką jego budowę, aby  był przystosowany do obsługi wielu osiedli mieszkaniowych równocześnie. 
Nie mniej ważne było, aby aplikacja obsługiwała funkcjonalności istotne zarówno z perspektywy mieszkańca, jak i administratora osiedla, łącząc w jednym systemie całość zagadnień dotyczących budynku. 
System zaprojektowano w ten sposób, aby jego działanie było intuicyjne, przejrzyste i łatwe nawet dla niedoświadczonego użytkownika, uwzględniając różny stopień umiejętności informatycznych mieszkańców budynku.
Dla najefektywniejszej realizacji powyższych celów przyjęto, że oprogramowanie powinno być stworzone w środowisku skonteneryzowanym z hybrydową aplikacją internetową.
\subsection{Wprowadzenie do tematyki}
W dzisiejszym  społeczeństwie obserwujemy coraz większą potrzebę automatyzacji wszelkich procesów, możliwości użycia aplikacji w miejsce standardowej korespondencji, w tym również elastyczną możliwość komunikacji, niezależną od godzin pracy biur i urzędów. Zauważono, że jedną z takich dziedzin jest możliwość łatwego i szybkiego dokonywania różnych zgłoszeń przez mieszkańców budynku czy osiedla mieszkaniowego do administracji, uwzględniając drobne usterki, ale również poważniejsze awarie czy zgłoszenia o charakterze ogólnym związanym z zarządzaniem budynkiem. 

Projekt aplikacji powstał w celu zastąpienia osobistej wizyty w  administracji spółdzielni, w godzinach jej pracy i możliwość dokonania dowolnego zgłoszenia poprzez wygodne, nieskomplikowane i niezależne od pory dnia działanie w systemie. Jednocześnie, informacje i zawiadomienia przygotowane przez administrację dla mieszkańców będą dostępne nie tylko na tablicy ogłoszeń w wersji papierowej, ale także w sekcji ogłoszeń w aplikacji, tak aby każdy mógł się z nimi zapoznać w dogodnej formie. 

Projektowane rozwiązanie systemu zgłaszania wniosków i usterek będzie korzystne również dla osób z niepełnosprawnościami, zwiększając ich komfort i wygodę komunikacji, bez konieczności przemieszczania się do biura administracji. Uwzględniono również usługę dla osób niedowidzących, które za pomocą dedykowanego oprogramowania mogą odsłuchać treść aktualnych ogłoszeń administracyjnych.

Powyższe argumenty przedstawiają sprawność komunikacji mieszkańców z administracją w zakresie zgłaszania szkód, usterek czy awarii oraz zapoznawania się z aktualnościami. 

Warto zauważyć, że benefitem projektowanego systemu jest także połączenie dokonanych zgłoszeń z przypisaniem ich realizacji poszczególnym pracownikom, sprawność komunikacji wewnątrz administracji, a tym samym zwiększenie efektywności działań, skrócenie czasu reakcji i podniesienie sprawności działania obsługi konserwatorskiej.
\subsection{Analiza istniejących rozwiązań}
Na rynku mieszkaniowym istnieje wiele systemów zajmujących się podobną tematyką, aczkolwiek większość z nich zajmuje się obsługą spółdzielni jedynie od strony kadrowej, co jest również bardzo istotnym zagadnieniem, ale nie obejmującym obsługi mieszkańców. W odróżnieniu od dostępnych rozwiązań,  proponowane w tej pracy rozwiązanie koncentruje się  na obsłudze kontaktu mieszkaniec - administracja jak również zarządzaniu pracownikami realizującymi zlecenia. Poniżej przedstawione są niektóre z nich: 
\begin{itemize}
    \item Moja Spółdzielnia
    \item PROBIT
    \item System ADA
    \item emieszkaniec.pl
    \item imieszkaniec.pl
    \item mmsoft - lokalnet
    \item sacer
    \item mieszczanin.pl
\end{itemize}