%%%%%%%%%%%%%%%%%%%%%%%%%%%%%%%%%%%%%%%%%%%%%%%%%%%%%%%
%% Bachelor's & Master's Thesis Template             %%
%% Copyleft by Artur M. Brodzki & Piotr Woźniak      %%
%% Faculty of Electronics and Information Technology %%
%% Warsaw University of Technology, 2019-2020        %%
%%%%%%%%%%%%%%%%%%%%%%%%%%%%%%%%%%%%%%%%%%%%%%%%%%%%%%%

\documentclass[
    left=2.5cm,         % Sadly, generic margin parameter
    right=2.5cm,        % doesnt't work, as it is
    top=2.5cm,          % superseded by more specific
    bottom=3cm,         % left...bottom parameters.
    bindingoffset=6mm,  % Optional binding offset.
    nohyphenation=false % You may turn off hyphenation, if don't like.
]{eiti/eiti-thesis}

\langpol % Dla języka angielskiego mamy \langeng
\graphicspath{{img/}}             % Katalog z obrazkami.
\addbibresource{bibliografia.bib} % Plik .bib z bibliografią

\begin{document}

%--------------------------------------
% Strona tytułowa
%--------------------------------------
\EngineerThesis % Dla pracy inżynierskiej mamy \EngineerThesis
\instytut{Informatyki i automatyki stosowanej}
\kierunek{Informatyka}
\specjalnosc{Inżynieria oprogramowania}
\title{
    Projekt i realizacja systemu do zgłaszania usterek w lokalach mieszkalnych.
}
\engtitle{ % Tytuł po angielsku do angielskiego streszczenia
    Design and implementation of a system for reporting faults in residential premises. 
}
\author{Krzysztof Fijałkowski}
\album{318652}
\promotor{dr Szymon Kijas}
\date{\the\year}
\maketitle

%--------------------------------------
% Streszczenie po polsku
%--------------------------------------
\cleardoublepage % Zaczynamy od nieparzystej strony
\streszczenie \lipsum[1-3]
\slowakluczowe XXX, XXX, XXX

%--------------------------------------
% Streszczenie po angielsku
%--------------------------------------
\newpage
\abstract \kant[1-3]
\keywords XXX, XXX, XXX

%--------------------------------------
% Oświadczenie o autorstwie
%--------------------------------------
\cleardoublepage  % Zaczynamy od nieparzystej strony
\pagestyle{plain}
\makeauthorship

%--------------------------------------
% Spis treści
%--------------------------------------
\cleardoublepage % Zaczynamy od nieparzystej strony
\tableofcontents

%--------------------------------------
% Rozdziały
%--------------------------------------
\cleardoublepage % Zaczynamy od nieparzystej strony
\pagestyle{headings}

\newpage % Rozdziały zaczynamy od nowej strony.
\section{Wprowadzenie}
\subsection{Cel pracy}
\subsection{Wprowadzenie do tematyki}
\subsection{Analiza istniejących rozwiązań}

\input{tex/2-analiza-wymagań}
\newpage % Rozdziały zaczynamy od nowej strony.
\section{Metodologia wytwarzania}
\newpage % Rozdziały zaczynamy od nowej strony.
\section{Definicja architektury}
\newpage
\section{Wdrożenie}
\input{tex/6-dane-trwałe}
\input{tex/7-interfejs-użytkownika}
\newpage
\section{Instrukcja użytkownika}
                            

%--------------------------------------------
% Literatura
%--------------------------------------------
\cleardoublepage % Zaczynamy od nieparzystej strony
\printbibliography

%--------------------------------------------
% Spisy (opcjonalne)
%--------------------------------------------
\newpage
\pagestyle{plain}

% Wykaz symboli i skrótów.
% Pamiętaj, żeby posortować symbole alfabetycznie
% we własnym zakresie. Ponieważ mało kto używa takiego wykazu,
% uznałem, że robienie automatycznie sortowanej listy
% na poziomie LaTeXa to za duży overkill.
% Makro \acronymlist generuje właściwy tytuł sekcji,
% w zależności od języka.
% Makro \acronym dodaje skrót/symbol do listy,
% zapewniając podstawowe formatowanie.
% //AB
\vspace{0.8cm}
\acronymlist
\acronym{EiTI}{Wydział Elektroniki i Technik Informacyjnych}
\acronym{PW}{Politechnika Warszawska}
\acronym{WEIRD}{ang. \emph{Western, Educated, Industrialized, Rich and Democratic}}

\listoffigurestoc     % Spis rysunków.
\vspace{1cm}          % vertical space
\listoftablestoc      % Spis tabel.
\vspace{1cm}          % vertical space
\listofappendicestoc  % Spis załączników

% Używając powyższych spisów jako szablonu,
% możesz tu dodać swój własny wykaz bądź listę,
% np. spis algorytmów.

\end{document} % Dobranoc.
